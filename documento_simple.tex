\documentclass[11pt,a4paper]{article}
\usepackage[utf8]{inputenc}
\usepackage[T1]{fontenc}
\usepackage[spanish]{babel}
\usepackage{graphicx}
\usepackage{xcolor}
\usepackage{geometry}

% Definir colores corporativos
\definecolor{maincolor}{RGB}{0, 75, 145}  % Azul corporativo
\definecolor{secondcolor}{RGB}{220, 53, 69}  % Rojo para alertas/acciones
\definecolor{thirdcolor}{RGB}{40, 167, 69}  % Verde para éxito

% Configuración de márgenes
\geometry{left=2.5cm,right=2.5cm,top=2.5cm,bottom=2.5cm}

\begin{document}

\begin{center}
\huge\textbf{\textcolor{maincolor}{FireGuardian}}\\[0.5cm]
\Large Sistema Web de Gestión y Consulta de Extintores\\[1cm]
\large\textbf{Documento Técnico}\\[2cm]

\normalsize
\textbf{Equipo YCC Extintores}\\[0.3cm]
Gómez Mendoza Jorman Daniel\\
Cetina Pool Raúl Abel\\
Quiñones Sánchez Christopher Germaine\\[1cm]

\textit{Presentado a:}\\
Mtra. Sara Jeannette Carrillo Ruiz\\[2cm]

\today
\end{center}

\newpage

\section{Descripción del Problema}

En el ámbito de la seguridad contra incendios, la gestión eficiente de extintores es fundamental para garantizar la protección adecuada en cualquier instalación. Sin embargo, muchas organizaciones enfrentan desafíos significativos en este aspecto:

\begin{itemize}
    \item \textbf{Seguimiento manual y propenso a errores:} Los métodos tradicionales basados en papel o hojas de cálculo son ineficientes y susceptibles a errores humanos.
    \item \textbf{Dificultad para mantener registros actualizados:} La información sobre ubicaciones, fechas de mantenimiento y vencimiento de extintores suele estar desactualizada.
    \item \textbf{Falta de acceso inmediato a la información:} Los técnicos en campo no pueden consultar o actualizar datos en tiempo real.
\end{itemize}

FireGuardian surge como solución a estos problemas, ofreciendo una plataforma web integral para la gestión de extintores.

\section{Diseño de la Arquitectura de Información}

FireGuardian implementa una arquitectura de información jerárquica y relacional que organiza los datos de manera lógica y accesible para diferentes tipos de usuarios.

\end{document}
